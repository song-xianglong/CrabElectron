\documentclass[9pt, mathserif]{beamer}
\usepackage{ctex}
\usefonttheme{serif}
\usetheme{Berlin}
\usecolortheme{seagull}
\usepackage[utf8]{inputenc}
\usepackage{amsfonts}
\usepackage{lmodern}
\usepackage{amsmath}
\usepackage{geometry}
\usepackage{graphicx}
\usepackage{tikz}
\usepackage{url}
\usepackage{geometry}
\usepackage{bm}
\usepackage{physics}
\usepackage{float}
\usepackage{subfigure}
\usepackage{wrapfig}
\usepackage{multirow}
\usepackage{xcolor}
\usepackage{indentfirst}
\setlength{\parindent}{2em}


\title{\textbf{\textbf{}}}
\author{\textbf{宋相龙\quad Xianglong Song}}
\institute{Boling Class of Physics, School of Physics, Nankai University, Tianjin 300071, China}
\date{Jan 25, 2024}

\begin{document}
    \begin{frame}
        \titlepage
    \end{frame}
    \begin{frame}
		\frametitle{Contents} 
		\tableofcontents
	\end{frame}
    \section{Crab Nebula}
        \begin{frame}
            \frametitle{Introduction}
            
        \end{frame}

    \section{References}
        \begin{frame}
            \begin{thebibliography}{1}
                \bibitem{bib1}
                {\it{高立模}}. 近代物理实验[J]. 南开大学出版社, 2006.
            \end{thebibliography}
        \end{frame}
\end{document}


